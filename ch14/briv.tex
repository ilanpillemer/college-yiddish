\documentclass{article}
%\usepackage[margin=2in]{geometry}
\usepackage[osf,p]{libertinus}
\usepackage{microtype}
\usepackage[pdfusetitle,hidelinks]{hyperref}

\usepackage[series={A,B,C}]{reledmac}
\usepackage{reledpar}

\usepackage{graphicx}
\usepackage{polyglossia}
\setmainlanguage{english}
\setotherlanguage{hebrew}
\gappto\captionshebrew{\renewcommand\chaptername{קאַפּיטל}}
\usepackage{metalogo}


%%linenumincrement*{1}
%%\firstlinenum*{1}
%%\setlength{\Lcolwidth}{0.44\textwidth}
%%\setlength{\Rcolwidth}{0.44\textwidth}

\begin{document}
%%\maxhnotesA{0.8\textheight}
\renewcommand{\abstractname}{\vspace{-\baselineskip}}
\title{A letter from Europe, 1948.}
\author{Transl. Ilan Pillemer}
\date{\today}

\maketitle
\abstract{
Chapter 14, College Yiddish, Translation exercise of a letter from Europe in 1948.
}
\newline
\begin{pairs}

\begin{Rightside}

\begin{RTL}
\begin{hebrew}
\beginnumbering

\autopar
"מיר זײַנען שוין דאָ אין דײַטשלאַנד דרײַ יאָר, און מיר קענען נאָך אַלץ פֿון דאַנען ניט אַװעקפֿאַרן.
איך האָב געהאָפֿט צו קענען פֿאָרן קײן ארץ-ישׂראל צו מײַן שװעסטער חנחן.
דו װײסט דאָך, זי איז שוין דאַרטן פֿון פֿאַר דער מלחמה, צוזאַמען מיט איר מאַן.
אָבער איצט לאָזט מען אַהין ניט פֿאָרן.
קײַן אַמעריקע קענען מיר אויך ניט פֿאַרן, װײַל איך און מײַן פֿרוי זײַנען אויף דער פּוילישער קװאָטע, און מיר דאַרפֿן װאַרטן אויף אַן אַמעריקאַנער װיזע אַ סך יאָרן.
מוזן מיר דערװײַל בלײַבן דאָ אין דײַטשלאַנד, צװישן מענטשן װאָס האָבן אומגעברענגט אָדער געהאָלפֿן אומברענגען אונדזערע ברידער.
אפֿט טראַכטן מיר, אַז די װעלט האָט אַונדז פֿאַרגעסן. דער סוף פֿון אונדזערע צרות איז נאָך אַזוי װײַט!


"מײַן ברודער יוסף האָבן די דײַטשן אומגעברענגט, און מיר װײסן אַפֿילו ניט, װען זײַן יאָרצײַט איז.
צוזאַמען מיט אונדז װוינט איצט מײַן אַנדערע ברודער אַבֿרהם.
די דײַטשן האָבן אים גענומען אויף אַרבעט צוזאַמען מיט נאָך פֿיר טויזנט מענער פֿון װאַרשעװער געטאָ.
זײ זײַנען געװען אין אַ קאָנצענטראַציע-לאַגער און האָבן געבויט װעגן.
פֿאַרן סוף פֿון דער מלחמה זײַנען זײ אָנטלאָפֿן פֿון קאָנצענטראַציע-לאַגער און האָבן זיך באַהאַלטן אין װאַלד.
דערװײַל האָבן אַבֿרהמס פֿרוי און זײַן טאָכטער זיך פֿאַרשטעלט פֿאַר קריסטן און האָבן זיך באַהאַלטן אין דעם ניט-ייִדישן טײל שטאָט.
זײ האָבן געאַרבעט װי דינסטן אין אַ פּויליש הויז.
קײנער האָט ניט געטאָרט װיסן, אַז זײ זײַנען ייִדן.
זײ האָבן געזען, װי די דײַטשן שטורעמען די געטאָ מיט טאַנקען און אַעראָפּלאַנען.
אין געטאָ האָבן געשלאָגן מענער, פֿרויען און קינדער; אָבער זײ אַלײן זײַנען געװען צװישן קריסטן און האָבן גאָרניט געקענט טאָן.


"נאָך דער באַפֿרײַונג האָט אַבֿרהם געפֿונען זײַן פֿרוי און טאָכטער.
װי דורך אַ נס זײַנען אַלע דרײַ געבליבן לעבן.
נאָך צװײַ יאָר זײַנען זײ געקומען אַהער.

\endnumbering
\end{hebrew}
\end{RTL}
\end{Rightside}


\begin{Leftside}
\begin{english}
\beginnumbering

\autopar
We have been already here in Germany for three years, and we still cannot leave from here.
I had hoped to be able to travel to the land of Israel - to my sister, Hannah.
As you well know, she was already there from before the war, together with her husband.
But currently they won't let anyone travel there.
We also cannot go to America because my wife and I are included in the Polish quota - and we need to wait many years for an American visa.
We must remain meanwhile here in Germany, amongst people who murdered, or helped murder, our brothers.
Often we think that the world has forgotten about us. The end of our woes is still so far away!


The Germans murdered my brother Yosef, and we do not even know when his Yartzheit is.
 At the moment, my other brother Avraham is living together with us.
 The Germans took him for labour, together with a further four thousand men from the Warsaw ghetto.
 They were in a concentration camp, and built roads.
 Before the end of the war, they escaped from the concentration camp and hid themselves in the forest.
 Meanwhile Abraham's wife and his daughter disguised themselves as Christians and hid themselves in the Gentile part of the city.
 They worked as maids in a Polish household.
 No-one could be allowed to know that they were Jews.
 They saw how the Germans stormed the ghetto with tanks and aeroplanes.
 In the ghetto men, women and children battled; but they were amongst the Christians and could not do anything themselves.
 
 
 After the liberation Avraham found his wife and daughter.
 Through a miracle all three of them survived.
 They came here two years ago.
 
\endnumbering
\end{english}
\end{Leftside}

\end{pairs}
\Columns


\end{document}
