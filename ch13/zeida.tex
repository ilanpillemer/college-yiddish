\documentclass{article}
\usepackage[osf,p]{libertinus}
\usepackage{microtype}
\usepackage[pdfusetitle,hidelinks]{hyperref}
\usepackage[series={},nocritical,noend,nofamiliar,noledgroup]{reledmac}
\usepackage{reledpar}

\usepackage{graphicx}
\usepackage{polyglossia}
\setmainlanguage{english}
\setotherlanguage{hebrew}
\gappto\captionshebrew{\renewcommand\chaptername{קאַפּיטל}}
\usepackage{metalogo}

\linenumincrement*{1}
\firstlinenum*{1}
\setlength{\Lcolwidth}{0.44\textwidth}
\setlength{\Rcolwidth}{0.44\textwidth}

\begin{document}
\renewcommand{\abstractname}{\vspace{-\baselineskip}}
\title{Extract from Ch13 College Yiddish}
\author{Translated by \\ Ilan Pillemer <ilan.pillemer@gmail.com>, \\ Alex O'Connor }
\date{\today}

\maketitle
\abstract{Warm up translation exercise assigned for week 1}
\newline
\begin{pairs}

\begin{Rightside}
\begin{RTL}
\begin{hebrew}
\beginnumbering


\pstart
״אַ חדר איז אַ טראַדיציאָנעלע שול, װוּ מען לערנט חומש מיט רשי, גמרא און אַלץ װאָס אַ ייִנגל דאַרף װיסן צו זײַן אַ פֿרומיר ייִד.
דער חדר איז געװען בײַם רבין אין דער הײם. דער רבי איז געװען זײער אַן אַרעמער און זײַן הויז זײַער אַ קלײַנס. האָבן מיר אַלע געלערנט אַרום אַ קלײַנעם טיש. אין חדר בין איך געגאַנגען ביז איך בין געװאָרן בר-מיצװה.
 נאָך דעם איז מײַן טאַטע געשטאָרבן, און איך האָב געדאַרפֿט אָנהײבן צו אַרבעטן. אַן אײַנבינדער האָט מיך גענומען צו זיך, און איך בין געװאָרן אַ לערנייִנגל בײַ אים."
\pend
\pstart
"און פֿאַר װאָס ביסטו אַװעקגעפֿאַרן פֿון זשיטאָמיר?"

\pend
\pstart
"אין רוסלאַנד איז געקומען זײער אַ שלעכטע צײַט. מען האָט געמאַכט פּאָגראָמען אויף ייִדן, און די אָרעמקײט איז געװען אַ גרויסע.
מײַן ברודער איז שוין דעמאָלט געװען אין אַמעריקע.

 װעגן אַמעריקע האָט מען דערצײלט, אַז דאָרטן װאַלגערט זיך גאָלד אויף די גאַסן.
מען האָט גערעדט װעגן שיכפּוצערס װאָס האָבן זיך אַרויפֿגעאַרבעט און זײַנען געװאָרן מיליאָנערן. האָב איך באַשלאָסן אַװעקצופֿאָרן צו מײַן ברודער."

\pend
\endnumbering
\end{hebrew}
\end{RTL}
\end{Rightside}




\begin{Leftside}
\begin{english}
\beginnumbering
\pstart
A cheder is a traditional school where one learns Chumash with Rashi, Gemara\footnotemark[1], and everything that a boy needs know to be an observant Jew.
The cheder took place at the teacher's home. The teacher was really impoverished, and his house was tiny\footnotemark[2].

Consequently\footnotemark[3], we all learnt around a small table. I went to cheder until I became bar-mitzvah. After this my father passed away, and I needed to begin to work. 
A book binder took me on, and I became his apprentice."
\pend
\pstart
 And why did you leave Zhitomir?
\pend
\pstart
Within Russia a really terrible era had arrived. Jews were suffering from pogroms.\footnotemark[4] And poverty had become widespread\footnotemark[5]. My brother, at that time, was already in America.
It was said about America that over there gold could be found on the streets. People spoke about shoe shiners who had worked themselves up and become millionaires. Therefore\footnotemark[6], I decided to emigrate to my brother.
\pend
 \footnotetext[1]{Chumash, the Pentateuch, is always taught with Rashi, the famous French exegist commentary, as is the Gemara, the Talmud. The Chumash contains the founding mythologies, and the Talmud the ideas and jurisprudence that determine Jewish communal laws and religious observances.} 
 \footnotetext[2]{the adjective of the teacher and his house follow the noun and this construction results in a repetition of the article, see 5.8 in "Grammar of the Yiddish Language" by D. Katz. He says this is more common in narrative styles. Further to this it could be translated into English if you want to literally keep the phrasing similar to "the teacher was a very impoverished one" and similarly for the room. But that is not idiomatic English.}
\footnotetext[3]{consecutive word order in the Yiddish, indicates consequence} 
\footnotetext[4]{violent murderous riots from mobs}
\footnotetext[5]{literally the word `large' is used}
\footnotetext[6]{consecutive word order in the Yiddish, indicates consequence} 
\endnumbering
\end{english}
\end{Leftside}

\end{pairs}
\Columns


\end{document}
